\documentclass[12pt, oneside]{article}
\usepackage[utf8]{inputenc}
\usepackage[shortlabels]{enumitem}
\usepackage[a4paper, total={6in, 8in}]{geometry}
\usepackage{ulem}
\usepackage[slovak]{babel}
\usepackage[T1]{fontenc}
\usepackage{caption}
\usepackage{subcaption}
\usepackage{hyperref}
\hypersetup{
    colorlinks=true,
    linkcolor=black,
    filecolor=magenta,      
    urlcolor=black,
}

\renewcommand{\familydefault}{\sfdefault}
\usepackage{graphicx}

\begin{document}

\thispagestyle{empty}
\begin{center}
\begin{minipage}{0.75\linewidth}
\centering
\uppercase{Slovenská Technická univerzita v Bratislave}
Fakulta informatiky a informačných technológií\\
Ilkovičova 2, 842 16 Bratislava 4\\
\vspace{7cm}
{\Large \textbf{Zadanie 1 - SIP Proxy} \par}
\vspace{7cm}
\begin{flushright}
{\Large Martin Nemec \\ FIIT STU \\ MTAA\\ Cvičenie: Štvrtok 8:00\\}
\end{flushright}
\end{minipage}
\end{center}

\newpage

\section{Implementácia}
Ako prostredie som si vybral python, kde som celé zadanie realizoval. 
Implementáciu proxy som čerpal z github repozitára: \textbf{\href{https://github.com/tirfil/PySipFullProxy}{PySipFullProxy}} 
\newline
Kód v repozitáry bolo potrebné opraviť, keďže bol písaný v staršej verzii Pythonu.
Po opravení množstva chýb som vytvoril vlastnú main funkciu, z ktorej som program spúšťal. Po sprevádzkovaní proxy serveru som bol schopný vykonať hovor. Ako SIP client som využíval Linphone na PC a tak isto aj na mobilnom telefóne.
\section{Knižnice}
Zoznam používaných knižníc:
\begin{itemize}
  \item socketserver
  \item re
  \item time
  \item logging
  \item socket
  \item sys
\end{itemize}

\section{Funkcionalita}
Z časti povinných bodov funkcionality je všetko funkčné a zaznamenané v pcap súboroch na mojom github repozitáry.
\newline
Z doplnkových funkcionalít sa mi podarilo implementovať tieto:
\begin{itemize}
  \item zrealizovanie konferenčného hovoru
  \item presmerovanie hovoru
  \item realizovanie videohovoru
  \item úprava SIP stavových kódov v zdrojovom kóde, kde som upravoval z angličtiny do slovenčiny
\end{itemize}
Tieto body su taktiež zaznamenané v pcap súboroch.

\section{Opis funkčnosti hovoru}
\subsection{Registrácia}
Pri registrácii sa pošle request REGISTER s IP adresou. Po úspešnej registrácii sa pošle 200 Pozitívne.
\subsection{Vytáčanie}
Po registrácii učastníka je možnosť zavolať na iné zariadenie. Po zavolaní následuje INVITE na daného uživateľa.
Následuje 100 Trying frame, kde sa pokúša spojiť, následne Ringing frame, kde začína zvoniť na druhom zariadení.
\subsection{Hovor}
Ak druhý účastník zdvihne hovor príde pozitívne ACK a taktiež príde frame s IP ďalšieho účastníka.
\subsection{Ukončenie hovoru}
Pri zrušení hovoru jednou stranou sa pošle BYE frame na adresu ďalšieho zariadenia na oznámenie ukončenia hovoru. 200 Ok pošle druhá strana naspäť a hovor skončil.

\section{Github}
\textbf{\href{https://github.com/nemo260/sip-mtaa}{Odkaz na môj github repozitár} }

\end{document}

